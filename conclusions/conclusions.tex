\chapter{Conclusions}
\label{chap:conclusions}

Results have been presented for a search of the \SM Higgs boson decaying into two photons at the \CMS experiment. An excess of events is observed over the background expectation with a local significance of $5.7\sigma$, where the \SM expectation is $5.2\sigma$, constituting a standalone discovery of the particle first observed by the ATLAS and \CMS experiments in 2012~\cite{ATLASDiscovery,CMSDiscovery}. The analysis strategy is to split events into a number of non-overlapping categories in order to increase the sensitivity to a signal and reduce the errors on measurements of the signal's couplings. The observed signal strength of the particle, relative to the \SM Higgs boson prediction, is $\sigma/\sigma_{SM} = 1.14^{+0.26}_{-0.23} \;\Bigl[\;^{+0.21}_{-0.21} \mathrm{(stat)} ^{+0.13}_{-0.09} \mathrm{(theory)} ^{+0.09}_{-0.05} \mathrm{(syst)} \Bigr]$. The observed excess is more apparent in the 7~TeV dataset in which the signal strength is found to be $\musm=2.22^{+0.60}_{-0.54}$, compared to a value of $\musm=0.90^{+0.25}_{-0.23}$ for the 8~TeV dataset. When forcing the signal in the two seperate datasets to have the same mass the compatibility between these two measurements is $1.9\sigma$. The observed mass of the particle is, $\;m_{H} = 124.72 \pm 0.35 \; \Bigl[ 0.31 \mathrm{\;(stat)} \pm 0.16 \mathrm{\;(syst)} \Bigr] \mathrm{\;GeV}$. To probe the coupling of the observed state to fermions and bosons, relative to the \SM, the signal strength is reparametrised to consist of two seperate components, with a common mass, namely that from \ggH and \ttH production, \RF, and from \VBF and \VH production, \RV. The observation of these parameters is $\RF=1.13^{+0.37}_{-0.32}$ and $\RV=1.16^{+0.58}_{-0.60}$ showing a high level of compatibility with the \SM prediction. The observed compatibility between the signal strength from each of the seperate production modes, \ggH, \VBF, \VH and \ttH, is found to have a probability, $p(\chi^{2})=49\%$. The observed compatbility between all of the analysis categories is found to have a probability, $p(\chi^{2})=74\%$. An analysis constructed to study the spin of the observed state is found to show consistency with the \SM at a level of $\p(\chi^{2})=86\%$. A spin-2 graviton produced entirely by gluon fusion is excluded at 94\%~C.L.~(where 92\% is expected) and a spin-2 graviton produced entirely by quark-antiquark annihilation is excluded at 85\%~C.L.~(where 83\% is expected).

In summary there is a clear observation of the new state in this channel, and it is found to be very compatible with the \SM Higgs. Futhermore, many studies of this particle's properties, since its discovery, in other decay modes and at other experiments suggest similar agreement with the \SM prediction. Whilst there is strong theoretical motivation for the existence of a Higgs-like state somewhere in the low mass region (see Chapter~\ref{chap:theory}) it is quite remarkable that many of the results shown here and elsewhere are compatible with preditions made several decades ago. Perhaps this is not the most interesting configuration of nature we could have hoped to see at the \LHC. It is beyond doubt that there is physics beyond the \SM; the Higgs self-coupling is quadratically divergent, neutrinos are not massless, the universe is known to consit of more than just the matter fermions discussed in Sec.~\ref{sec:standardmodel} with a plethora of evidence for dark matter and dark energy, the lack of anti-matter in the universe does not fit with our expectation from creation and annihilation and the difference is not accounted for by current measurements of charge-parity violation, etc. There are theories which can explain some of these features, and many detectors looking for direct experimental signatures of them, but discussion of these is beyond the scope of this thesis. However, it is apparent that more detailed study of the Higgs sector could provide insight. Any deviations from the \SM, probed by precise measurements of the observed bosons couplings, could provide evidence for new physics. Whether through the loop in \Hgg decays or via decay chains such as $X\rightarrow H\rightarrow Y$ for a new particle $X$ and known decay products $Y$. Consequently, it is really only the beginning of Higgs physics. The discovery has happened and now the focus must turn to precision measurements of the particle's properties in the hope that it points the way to a new sector of unification in particle physics.

\hfill\textit{Matthew Kenzie}
