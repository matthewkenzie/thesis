\chapter{Common analysis components}
\label{chap:common_analysis_components}
\chapterquote{I don't have a quote}
{Matthew Kenzie, 1785--1854}

\section{Description}

A description of the analysis components and pre-requisites for a Higgs to two photon search. There would be some explanation of boosted decision trees. Specifics in identifying photons, measuring their energies and the per photon energy resolution. A description of particle flow and its use as an isolation variable. Specifics of how tracks and converted photons are used to identify the primary vertex. Some description of jets, electrons, muons and MET which are used to tag exclusive Higgs production modes.

\textbf{20 pages}

\section{Bulk}

This thesis describes three complementary analysis regimes in the Higgs to two photons search at CMS. These differ in their photon selection, event selection, event classification (or categorisation) and statistical methods for extracting results. They are described in the following chapter (Chapter~\ref{chap:selection_and_categorisation}). However, there are many components which they share. These are detailed below.

As we have seen in Eq.~\ref{eq:invmass} the diphoton invariant mass is constructed from the two photon energies and the angle between them so clearly important considerations are good photon energy resolution, good position resolution (location they hit the detector) and good vertex resolution (the location of the primary interaction).


\section{Data samples}
\subsection{Trigger}
\subsection{Monte Carlo}
\subsection{Pileup}
\subsection{Beamspot}

\section{Energy measurement of photons}
\label{sec:photon_energy}

\section{Photon preselection}
\label{sec:photon_presel}

\section{Vertex reconstruction}
\label{sec:vtx_reco}

\section{Associated objects}
\subsection{Jets}
\subsection{Electrons}
\subsection{Muons}
\subsection{MET}

\section{Using \Zee decays for validation and effiency measurements}

