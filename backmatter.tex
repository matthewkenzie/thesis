%\begin{colophon}
  %This thesis was made in \LaTeXe{} using the ``hepthesis'' class~\cite{hepthesis}.
%\end{colophon}

%% You're recommended to use the eprint-aware biblio styles which
%% can be obtained from e.g. www.arxiv.org. The file mythesis.bib
%% is derived from the source using the SPIRES Bibtex service.
\bibliographystyle{h-physrev}
\bibliography{mythesis}

%% I prefer to put these tables here rather than making the
%% front matter seemingly interminable. No-one cares, anyway!
\listoffigures
\listoftables

%% If you have time and interest to generate a (decent) index,
%% then you've clearly spent more time on the write-up than the 
%% research ;-)
%\printindex

% declare acronyms
\newpage
\thispagestyle{empty}
\phantomsection
\addcontentsline{toc}{chapter}{List of Acronyms}
\vspace*{1.95cm} \hspace*{-0.155cm} %,88
\textbf{{\huge \sffamily List of Acronyms}\\}
\vspace*{0.5cm} 
\begin{acronym}[AAAAAAA]
\acro {bdt} [BDT] {Boosted Decision Tree}
\acro {bdts} [BDTs] {Boosted Decision Trees}
\acro {cern} [CERN] {European Organization for Nuclear Research}
\acro {cms} [CMS] {Compact Muon Solenoid}
\acro {ecal} [ECAL] {Electromagnetic calorimeter}
\acro {hcal} [HCAL] {Hadronic calorimeter}
\acro {lhc} [LHC] {Large Hadron Collider}
\acro {mva} [MVA] {Multivariate analysis}
\acro {mvas} [MVAs] {Multivariate analyses}
\acro {sm} [SM] {Standard Model}
\end{acronym}


