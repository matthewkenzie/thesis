\chapter{Selection and Categorisation}
\label{chap:selection_and_categorisation}
\chapterquote{Science never solves a problem without creating ten more.}
{George Bernard Shaw, 1856 -- 1950}

\section{Description}

A description of the selection and categorisation under two schemes � a robust cut-based driven approach (used later for the spin) and a highly optimized, standard model specific multivariate method (used later for the nominal analysis and the sideband background method). This would also include the validation procedure with Z->ee decays.

\textbf{15 pages}

\textbf{Brief outline of the three analyses being discussed and their uses}

This thesis presents results of three different analyses used in the Higgs to two photon search at CMS. The first, the \acf{CiC} analysis, is designed for simplicity and robustness as a cut based approach used to cross check the main result and, owing to it's low level of model dependence, for the spin analysis. The other two are more aggressive multivariate methods optimised specifically to search for a \ac{sm} Higgs boson. The first of these has a fully parametric definition of the diphoton invariant mass spectrum and as such is know as the \acf{MFM} or \acf{MBM} analysis. The second serves to cross check the background, the most significant unknown in a search like this, by extracting the background under the signal region from sidebands in the diphoton invariant mass spectrum, referred to as the \acf{SMVA}.

% ---- SECTION ----
\section{Photon selection}
\label{sec:pho_selection}
  
  \subsection{Selection using cuts in categories}
  \label{sec:cic}

  \subsection{Photon ID MVA}
  \label{sec:pho_id_mva}
  
  \subsection{Diphoton event level MVA}
  \label{sec:diphoton_bdt}

% ---- SECTION ----
\section{Event categorisation}
\label{sec:categorisation}

  \subsection{Exclusive mode tagging}
  \label{sec:exclusive_tags}

  \subsection{Inclusive mode categorisation in the mass factorised analysis}
  \label{sec:inclusive_cats_massfac}

  \subsection{Inclusive mode categorisation in the sideband analysis}
  \label{sec:invlusive_cats_sideband}

% ---- SECTION ----
\section{Signal modelling}
\label{sec:signal_model}

  \subsection{Mass factorised analysis}
  \subsection{Sideband analysis}

% ---- SECTION ----
\section{Background modelling}
\label{sec:background_model}

  \subsection{Mass factorised analysis}
  \subsection{Sideband analysis}
