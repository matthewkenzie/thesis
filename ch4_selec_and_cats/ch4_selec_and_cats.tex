\chapter{Selection and Categorisation}
\label{chap:selection_and_categorisation}
\chapterquote{I don't have a quote}
{Matthew Kenzie, 1785--1854}

\section{Description}

A description of the selection and categorisation under two schemes � a robust cut-based driven approach (used later for the spin) and a highly optimized, standard model specific multivariate method (used later for the nominal analysis and the sideband background method). This would also include the validation procedure with Z->ee decays.

\textbf{15 pages}

This thesis presents results of three different analyses used in the Higgs to two photon search at CMS. The first, the \acf{CiC} analysis, is designed for simplicity and robustness as a cut based approach used to cross check the main result and, owing to it's low level of model dependence, for the spin analysis. The other two are more aggressive multivariate methods optimised specifically to search for a \ac{sm} Higgs boson. The first of these has a fully parametric definition of the diphoton invariant mass spectrum and as such is know as the \acf{MFM} or \acf{MBM} analysis. The second serves to cross check the background, the most significant unknown in a search like this, by extracting the background under the signal region from sidebands in the diphoton invariant mass spectrum, referred to as the \acf{SMVA}.


\section{The cuts in categories analysis}
\label{sec:cic}
Photon identification is performed using a set a cuts separately optimised in different categories. Photons are split into four categories defined in terms of pseudorapidity (whether their supercluster position is in the barrel or end cap) and \rnine

\section{The mass factorized MVA analysis}
\label{sec:mva}

\section{The sideband MVA analysis}

\section{Efficiency corrections and validation with \Zee decays}
\label{sec:zee}
