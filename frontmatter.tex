%% Title
\titlepage[Imperial College London]%
{A dissertation submitted to Imperial College London\\
  for the degree of Doctor of Philosophy}

%% Abstract
\begin{abstract}%[\smaller \thetitle\\ \vspace*{1cm} \smaller {\theauthor}]
  %\thispagestyle{empty}
  Results are presented of a search for the \SM Higgs boson decaying into two photons at the \CMS experiment housed at the \LHC, CERN. An excess of events is observed over the background expectation with a local significance of $5.7\sigma$, where the \SM expectation is $5.2\sigma$, constituting a standalone discovery of the particle first observed by the ATLAS and \CMS experiments in July 2012. Measurements of the particles signal strength, mass and couplings are presented along with an analysis of its spin. The results show a high level of compatibility with the predictions for a \SM Higgs boson. The observed states signal strength relative to the \SM expectation is found to be $\musm=1.14^{+0.26}_{-0.23}$. The observed states mass is found to be $124.72\pm 0.35$~GeV. The respective couplings of the particle to fermions and bosons, relative to the \SM expectation, are found to be $\RF=1.13^{+0.37}_{-0.32}$ and $\RV=1.16^{+0.58}_{-0.60}$. A spin-2 graviton, produced entirely by gluon fusion, is excluded at 94\%~C.L.~(92\% expected) and a spin-2 graviton, produced entirely by quark-antiquark annihilation, is exlucded at 86\%~C.L.~(83\% expected). 
\end{abstract}


%% Declaration
\begin{declaration}
  This dissertation is the result of my own work, except where explicit
  reference is made to the work of others, and has not been submitted
  for another qualification to this or any other university. Figures from other
  sources are labelled as ``CMS" or ``CMS Preliminary" and the source is referenced in the figure captions.
  This dissertation does not exceed the word limit for the respective Degree
  Committee.
  \vspace*{1cm}
  \begin{flushright}
    Matthew Kenzie
  \end{flushright}
\end{declaration}


%% Acknowledgements
\begin{acknowledgements}
  Firstly I would like to thank my supervisor Paul Dauncey, mainly for the numerous meals, coffees and beers he has bought me but also for his exceptional academic mentoring, shared knowledge and personal support. Cheers for putting up with my laziness especially in administrative matters and paperwork. Thanks also to my other (unofficial) supervisor Chris Seez for his advice, both scientific and political, and for many stimulating discussions, even though I have still not been invited to share a cigar. I would also like to thank the CMS \Hgg group, some of whose hard fought analysis techniques and figures are used in this thesis. The many interesting discussions (arguments) and ideas have dictated my enjoyment of particle physics and desire to stay in the field.

  Thank you to all of my friends, who have endured my dreadful sense of humour and general lack of respect for punctuality, tidiness and organisation. I can't ``name check" everyone but to Darren, Patrick, Andrew and Indy - thanks for your time. To Emma, thank you so much for putting up with me, especially over long distance, and thanks for allowing me to follow my passion whilst still maintaining a (semi) functional relationship with you. Lastly, to my family. My brothers have helped me see the amusing side of ``a post-doc" and without the support of my parents I would simply not have had the opportunity to do this. For that, and all the other countless things, I will be eternally grateful. 

\end{acknowledgements}


%% Preface
%\begin{preface}
%  Blah blah blah
%\end{preface}

%% Strictly optional!
\frontquote%
  {For my father.}%
  {}

%% ToC
\tableofcontents
\listoffigures
\listoftables
\newpage
\thispagestyle{empty}
\phantomsection
\addcontentsline{toc}{chapter}{List of Acronyms}
\vspace*{1.95cm} \hspace*{-0.155cm} %,88
\textbf{{\huge List of Acronyms}\\}
\vspace*{0.5cm}
\begin{acronym}[AAAAAA]
\acro {BDT} [BDT] {Boosted Decision Tree}
\acro {BDTs} [BDTs] {Boosted Decision Trees}
\acro {CERN} [CERN] {European Organization for Nuclear Research}
\acro {CMS} [CMS] {Compact Muon Solenoid}
\acro {ECAL} [ECAL] {Electromagnetic calorimeter}
\acro {HCAL} [HCAL] {Hadronic calorimeter}
\acro {LHC} [LHC] {Large Hadron Collider}
\acro {MVA} [MVA] {Multivariate analysis}
\acro {MVAs} [MVAs] {Multivariate analyses}
\acro {CiC} [CiC] {Cuts in Categories}
\acro {MFM} [MFM] {Mass Factorized MVA}
\acro {MBM} [MBM] {Mass Blind MVA}
\acro {SMVA} [SMVA] {Sideband MVA}
\acro {SM} [SM] {Standard Model}
\acro {PbWO} [PbWO$_{4}$] {lead tungstate}
\acro {APD} [APD] {avalance photodiode}
\acro {APDs} [APDs] {avalance photodiodes}
\acro {VPTs} [VPTs] {vacuum phototriodes}
\acro {GED} [GED] {global event description}
\end{acronym}





