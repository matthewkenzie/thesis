%% Title
\titlepage[Imperial College London \\ Department of High Energy Physics]%
{A dissertation submitted to Imperial College London\\
  for the degree of Doctor of Philosophy}

%% Abstract
\begin{abstract}%[\smaller \thetitle\\ \vspace*{1cm} \smaller {\theauthor}]
  %\thispagestyle{empty}
  Results are presented of a search for the \acf{SM} Higgs boson decaying into two photons at the \acf{CMS} experiment housed at the \acf{LHC}, CERN. An excess of events is observed over the background expectation with a local significance of $5.7\sigma$, where the \ac{SM} expectation is $5.2\sigma$, constituting a standalone discovery of the particle first observed by the ATLAS and \CMS experiments in July 2012. Measurements of the particle's signal strength, mass and couplings are presented along with an analysis of its spin. The results show a high level of compatibility with the predictions for a \SM Higgs boson. The observed state's signal strength relative to the \SM expectation is found to be $\musm=1.14^{+0.26}_{-0.23}$. The observed state's mass is found to be $124.72\pm 0.35$~GeV. The signal strength relative to the \SM expectation when probing production mechanisms through fermionic modes only is $1.13^{+0.37}_{-0.31}$, and from bosonic production modes only is $1.16^{+0.63}_{-0.57}$. A spin-2 graviton, produced entirely by gluon fusion, is excluded at 94\%~C.L.~(92\% expected) and a spin-2 graviton, produced entirely by quark-antiquark annihilation, is excluded at 85\%~C.L.~(83\% expected).
\end{abstract}


%% Declaration
\begin{declaration}
  This dissertation is not the result of entirely my own work. The concepts and ideas described in Chapters~\ref{chap:theory} and~\ref{chap:cms}, whilst my own words, are based on the work of others. Considerable parts of Chapters~\ref{chap:common_analysis_components} and~\ref{chap:selection_and_categorisation}, specifically Sections~\ref{sec:photon_energy},~\ref{sec:vtx_reco},~\ref{sec:photon_presel} and~\ref{sec:event_selection}, are produced in collaboration with other members of the \acs{CMS} \Hgg group and wider \acs{CMS} collaboration. Consequently some of the techniques and ideas presented are not entirely my own. The majority of everything presented in Chapters~\ref{chap:analysis} and~\ref{chap:results} is my own work. Where appropriate ideas from others are referenced and figures from other sources are labelled with ``\Hgg", ``LHC" or ``CMS" and the source is referenced in the figure captions.

  This dissertation has not been submitted for another qualification to this or any other university and does not exceed the word limit for the respective Degree Committee.

The copyright of this thesis rests with the author and is made available under a Creative Commons Attribution Non-Commercial No Derivatives licence. Researchers are free to copy, distribute or transmit the thesis on the condition that they attribute it, that they do not use it for commercial purposes and that they do not alter, transform or build upon it. For any reuse or redistribution, researchers must make clear to others the licence terms of this work

  \vspace*{1cm}
  \begin{flushright}
    Matthew Kenzie
  \end{flushright}
\end{declaration}


%% Acknowledgements
\begin{acknowledgements}
  Firstly I would like to thank my supervisor Paul Dauncey, mainly for the numerous meals, coffees and beers he has bought me but also for his exceptional academic mentoring, shared knowledge and personal support. Cheers for putting up with my laziness especially in administrative matters and paperwork. Thanks also to my other (unofficial) supervisor Chris Seez for his advice, both scientific and political, and for many stimulating discussions, even though I have still not been invited to share a cigar! I owe a huge amount to Nick Wardle for his close collaboration and help over the years. I would also like to thank the CMS \Hgg group, some of whose hard fought analysis techniques and figures are used in this thesis. The many interesting discussions (arguments) and ideas have dictated my enjoyment of particle physics and desire to stay in the field. Thanks to the Science and Technology Facilities Council (STFC) and the Grundy Educational Trust for funding my PhD.

  Thank you to all of my friends, who have endured my dreadful sense of humour and general lack of respect for punctuality, tidiness and organisation. I can't ``name check" everyone but to Darren, Patrick and Andrew - thanks for your time. To Emma, thank you so much for putting up with me, especially over long distance, and thanks for allowing me to follow my passion whilst still maintaining a (semi) functional relationship with you. Lastly, to my family. My brothers have helped me see the amusing side of ``a post-doc" and without the support of my parents I would simply not have had the opportunity to do this. For that, and all the other countless things, I will be eternally grateful.

\end{acknowledgements}


%% Preface
%\begin{preface}
%  Blah blah blah
%\end{preface}

%% Strictly optional!
\frontquote{For my father}{}

\frontquote%
  {Data! data! data! I cannot make bricks without clay}
  {The Adventures of Sherlock Holmes, Sir Arthur Conan Doyle}

%% ToC
\tableofcontents
\listoffigures
\listoftables
\newpage
\addcontentsline{toc}{chapter}{List of Acronyms}
\chapter*{List of Acronyms}
\phantomsection
\begin{acronym}[AAAAAA]
\acro {BDT} [BDT] {Boosted Decision Tree}
\acro {BDTs} [BDTs] {Boosted Decision Trees}
\acro {CERN} [CERN] {European Organization for Nuclear Research}
\acro {CMS} [CMS] {Compact Muon Solenoid}
\acro {ECAL} [ECAL] {Electromagnetic calorimeter}
\acro {HCAL} [HCAL] {Hadronic calorimeter}
\acro {LHC} [LHC] {Large Hadron Collider}
\acro {MVA} [MVA] {Multivariate analysis}
\acro {MVAs} [MVAs] {Multivariate analyses}
\acro {CiC} [CiC] {Cuts in Categories}
\acro {MFM} [MFM] {Mass Factorized MVA}
\acro {MBM} [MBM] {Mass Blind MVA}
\acro {SMVA} [SMVA] {Sideband MVA}
\acro {SM} [SM] {Standard Model}
\acro {PbWO} [PbWO$_{4}$] {lead tungstate}
\acro {APD} [APD] {avalance photodiode}
\acro {APDs} [APDs] {avalance photodiodes}
\acro {VPTs} [VPTs] {vacuum phototriodes}
\acro {GED} [GED] {global event description}
\end{acronym}





\newpage
\thispagestyle{empty}
\mbox{}
