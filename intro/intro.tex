\chapter{Introduction}
\label{chap:intro}

The discovery of a new particle with a mass around 125~\GeV in the search for the \SM Higgs boson was announced simultaneously by the ATLAS and CMS collaborations in July 2012~\cite{ATLASDiscovery,CMSDiscovery}. Since then, about three times more data has been taken at the \LHC before the running period including 2011 and 2012 (referred to as ``Run 1") terminated in December 2012 for scheduled maintenance and upgrades. One of the primary goals of the Higgs groups at the \LHC is now to study the properties of this new resonance and determine whether it is the \SM Higgs boson or not. Some of the work in this thesis contributed towards the discovery in 2012 but much of the work detailed here builds upon that and focuses on measuring the properties of the new state in its decay into two photons. 

Chapter~\ref{chap:theory} gives a description of the fundamental constituents of matter and the forces that govern their interactions. The framework which underlies our theoretical predictions is known as the \SM of particle physics and the concepts behind the theory, in the context of local gauge symmetries, are explained. The mechanism by which the fundamental particles acquire a mass, spontaneous symmetry breaking (or alternatively the Higgs mechanism), is summarised and serves as a motivation for the existence of, and consequent desire to search for, the Higgs boson. The chapter concludes by discussing Higgs boson production at the \LHC and its decay into two photons with a focus on the predominant backgrounds for this search and how one can measure its spin.

In Chapter~\ref{chap:cms} the main apparatus for the analysis, the \CMS detector, is detailed. There is an explanation of the main detector subsystems with a particular focus on the \ECAL which is used to measure photon energies. A short description of some of the physics object reconstruction essentials is also given; particle flow, jets, isolation and pileup.

This thesis presents two complementary analyses and an additional analysis tailored to separate between different Higgs spin hypotheses. Chapter~\ref{chap:common_analysis_components} gives a description of the common analysis elements which are shared by all three. The topics covered include the datasets and \MC simulation, photon energy measurement and primary vertex location of the Higgs decay. There are also some other useful preliminary topics discussed; the use of \aclp{BDT} (\acs{BDT}) and the use of the \Zee decay as a control source for the \Hgg analyses.

Chapter~\ref{chap:selection_and_categorisation} gives a description of the event selection used in the three analyses and explains how the events are split into categories in order to improve sensitivity and help to reduce the errors on Higgs couplings measurements.

The full details of the statistical treatment of the data are explained in Chapter~\ref{chap:analysis}, which also includes a description of the signal and background modelling and the treatment of systematic uncertainties.

The results are presented and discussed in Chapter~\ref{chap:results} and there are final comments and conclusions in Chapter~\ref{chap:conclusions}.
